\documentclass[12pt]{iv003}
\usepackage[T1]{fontenc}
\usepackage[utf8]{inputenc}
\usepackage{lmodern}
\usepackage[czech]{babel}
\usepackage{amsmath, amsthm}
\usepackage{enumerate}

%% algoritmy
%{{{ Latex definitions
% ALGORITHM typesetting
\usepackage[noline,boxruled,longend,linesnumbered,czech]{algorithm2e}
%\SetKwInOut{Input}{vstup}\SetKwInOut{Output}{výstup}
\SetKwInput{Input}{vstup}
\SetKwInput{Output}{výstup}
\newcommand{\donothing}[1]{#1}
\SetArgSty{donothing}
\SetCommentSty{emph}
%\IncMargin{1em}
%\setlength{\algomargin}{1em}
\DontPrintSemicolon
\SetKwIF{If}{ElseIf}{Else}{if}{then}{else if}{else}{fi}
\SetKwFor{For}{for}{do}{od}
\SetKwFor{While}{while}{do}{od}

% hack to set the correct width of the algorithm
\usepackage{xpatch}
\xpretocmd{\algorithm}{\hsize=\linewidth}{}{}
\xpretocmd{\function}{\hsize=\linewidth}{}{}
\xpretocmd{\procedure}{\hsize=\linewidth}{}{}
%%/algoritmy
\usepackage{mathtools}
\DeclarePairedDelimiter\floor{\lfloor}{\rfloor}

%%% ZDE SADA, UKOL CISLO X, JMENO, UCO
\setexercise{1}
\setassignment{3}
\setstudenta{Karel Kubíček}
\setucoa{408351}
\setstudentb{Henrich Lauko}
\setucob{410438}


\begin{document}
Datovou strukturou, která splňuje podmínky stanovené zadáním je B+ strom se stupněm 4. Navíc oproti normálnímu B+ stromu podle zadání do uzlů přidáme hodnotu \textit{small[x]}.
\begin{enumerate}
	\item \textsc{Minimum} hledáme vždy v nelevějším podstromu. Rekurzivně se tedy volá minimum na nejlevější podstrom, dokud se nedosáhne úrovně listů, kde nejlevější potomek je minimum našeho stromu.\\
	\begin{procedure}[H]
	\SetKwFunction{ffminimum}{minimum}
		\caption{minimum($T$) }
		\Input{B+ strom $T$, ve kterém máme minimum hledat}
		\Output{ukazatel na minimim}
		\uIf{$T$ je list} {
			\Return{$T$} \tcp{nalezeno minimum, vracím ukazatel}
		}
		\uElse {
			\Return{\ffminimum($T.firstSubTree$)} \tcp{firstSubTree je nejlevější podstrom}
		}
	\end{procedure}
	\item \textsc{Insert}($k$) se skládá ze 2 částí. První částí je nalezení správného místa, na které prvek patří a zařazení na toto místo, druhou fází je v případě překročení limitu 4 potomci na strom rozdělení uzlu s 5 potomky na 2 uzly, jeden se dvěma, druhý se třema potomky a vypropagování kontroly velkého počtu potomků o úroveň výše.\\
	\begin{procedure}[H]
	\SetKwFunction{ffinsert}{insert}
		\caption{insert($T, k$) }
		\Input{B+ strom $T$, do kterého máme přidat klíč $k$}
		\Output{}
		\tcp{TODO}
	\end{procedure}
	\item \textsc{Decrease Key}($x,k$)
	\item \textsc{Delete}($x$)
	\item \textsc{Extract Min}
\end{enumerate}

\end{document}


















