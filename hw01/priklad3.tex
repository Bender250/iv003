\documentclass[12pt]{iv003}
\usepackage[T1]{fontenc}
\usepackage[utf8]{inputenc}
\usepackage{lmodern}
\usepackage[czech]{babel}
\usepackage{amsmath, amsthm}
\usepackage{enumerate}

%% algoritmy
%{{{ Latex definitions
% ALGORITHM typesetting
\usepackage[noline,boxruled,longend,linesnumbered,czech]{algorithm2e}
%\SetKwInOut{Input}{vstup}\SetKwInOut{Output}{výstup}
\SetKwInput{Input}{vstup}
\SetKwInput{Output}{výstup}
\newcommand{\donothing}[1]{#1}
\SetArgSty{donothing}
\SetCommentSty{emph}
%\IncMargin{1em}
%\setlength{\algomargin}{1em}
\DontPrintSemicolon
\SetKwIF{If}{ElseIf}{Else}{if}{then}{else if}{else}{fi}
\SetKwFor{For}{for}{do}{od}
\SetKwFor{While}{while}{do}{od}

% hack to set the correct width of the algorithm
\usepackage{xpatch}
\xpretocmd{\algorithm}{\hsize=\linewidth}{}{}
\xpretocmd{\function}{\hsize=\linewidth}{}{}
\xpretocmd{\procedure}{\hsize=\linewidth}{}{}
%%/algoritmy
\usepackage{mathtools}
\DeclarePairedDelimiter\floor{\lfloor}{\rfloor}

%%% ZDE SADA, UKOL CISLO X, JMENO, UCO
\setexercise{1}
\setassignment{3}
\setstudenta{Karel Kubíček}
\setucoa{408351}
\setstudentb{Henrich Lauko}
\setucob{410438}


\begin{document}
\SetKwFunction{ffselectkey}{selectKey}
\SetKwFunction{ffcontrolnode}{controlNode}

Datovou strukturou, která splňuje podmínky stanovené zadáním je B+ strom se stupněm 4.
Každý uzel bude mít tyto atributy: $p[x]$ je ukazatel na podstrom $x$, $pp$ je ukazatel na rodiče a $small[x]$ je klíč odpovídající nejmenší hodnotě v podstromu $x$.

\begin{enumerate}
	\item \textsc{Minimum} hledáme vždy v nejlevějším podstromu. Rekurzivně se tedy volá minimum na nejlevější podstrom, dokud se nedosáhne úrovně listů, kde nejlevější potomek je minimum našeho stromu.\\
	
	\begin{procedure}[H]
	\SetKwFunction{ffminimum}{minimum}
		\caption{minimum($T$) }
		\Input{B+ strom $T$, ve kterém máme minimum hledat}
		\Output{ukazatel na minimum}
		\uIf{$T$ je list} {
			\Return{$T$} \tcp{nalezeno minimum, vracím ukazatel}
		}
		\uElse {
			\Return{\ffminimum($T.p0$)} \tcp{firstSubTree je nejlevější podstrom}
		}
	\end{procedure}
	Časová složitost hledání minima je logaritmická, jelikož výška stromu je logaritmická (bude dokázáno níže u insert) a rekurze se v každém zanoření volá na strom s výškou o jedna menší.
	\item \textsc{Insert}($k$) se skládá ze 2 částí. První částí je nalezení správného místa, na které prvek patří a zařazení na toto místo, druhou fází je v případě překročení limitu 4 potomci na strom rozdělení uzlu s 5 potomky na 2 uzly, jeden se dvěma, druhý se třema potomky a vypropagování kontroly velkého počtu potomků o úroveň výše.\\
	
	\begin{procedure}[H]
	\SetKwFunction{ffinsert}{insert}
		\caption{insert($T, k$) }
		\Input{B+ strom $T$, do kterého máme přidat klíč $k$}
		\Output{}
		$x \leftarrow $ \ffselectkey($k$) \tcp{určí, do kterého podstromu patří zadaný klíč ($k > small[x] \wedge k \leq small[x+1]$)}
		\uIf{$T$ je list} {
			vlož $k$ na místo za $small[x]$ a posuň zbylé klíče \;
			\ffcontrolnode($T$) \;
		}
		\uElse {
			\ffinsert($T.p[x], k$) \;
		}
	\end{procedure}
	
	\begin{procedure}[H]
		\caption{controlNode($T$) }
		\Input{B+ strom $T$, jehož kořenový uzel kontrolujeme}
		\Output{}
		\uIf{počet klíčů v $T$ je 5} {
			rozděl uzel na uzel se 3 a 2 klíči \;
			vytvoř nový klíč v rodičovském uzlu \;
			\ffcontrolnode($T.pp$) \tcp{zkontroluj rodičovský uzel, pokud neexistuje, tak ho vytvoř}
		}
		\uIf{počet klíčů v $T$ je 1} {
			spoj uzel se sousedním uzlem s méně klíči \;
			odstraň z rodičovského uzlu klíč pro uzel $T$ \;
			\ffcontrolnode($T.pp.p[x]$) \tcp{zkontroluj uzel vzniklý spojením}
		}
	\end{procedure}
	I insert je logaritmické asymptotické časové složitosti. První část algoritmu udělá maximálně $\log_{2}(n)$ kroků při hledání správného listu, do kterého má klíč zařadit, následující procedura \ffcontrolnode v případě plné zaplněnosti všech rodičovských uzlů udělá maximálně $\log_{4}(n)$ (ale pokud by se dělení uzlů vypropagovalo až ke kořeni, pak by i první část musela proběhnout v $\log_{4}(n)$ krocích). To dohromady patří do $\mathcal{O}(\log(n))$
	\item \textsc{Delete}($x$)
	
	\begin{procedure}[H]
	\SetKwFunction{ffdelete}{delete}
		\caption{delete($T$) }
		\Input{B+ strom $T$, který máme vymazat}
		\Output{}
		odstraň z rodičovského uzlu klíč pro uzel $T$ \;
		\ffcontrolNode($T.pp$) \;
		uvolni paměť $T$ \;
	\end{procedure}
	
	Podobná procedura, jako insert, tentokrát nutno kontrolovat, jestli není v rodičovském uzlu klíčů málo (1). To, že operace patří do $\mathcal{O}(\log(n))$
	\item \textsc{Decrease Key}($x,k$) se dá realizovat posloupností procedur \ffdelete($x$) a následného \ffinsert($k$). To v součtu znamená složitost $2\cdot \log(n) \in \mathcal{O}(\log(n))$.
	\item \textsc{Extract Min}se dá realizovat složením procedur \ffdelete(\ffminimum). To v součtu znamená složitost $2\cdot \log(n) \in \mathcal{O}(\log(n))$.
\end{enumerate}
U důkazů složitosti jsem vždy předpokládal, že strom bude mít výšku $\log_{4}(n)$ až $\log_{4}(n)$. Pokud by tato vlastnost neplatila, pak by složitost operací neodpovídala zadání. Vlastnosti stromu tedy stojí na proceduře \ffcontrolnode.
\end{document}


















