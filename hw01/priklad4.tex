\documentclass[12pt]{iv003}
\usepackage[T1]{fontenc}
\usepackage[utf8]{inputenc}
\usepackage{lmodern}
\usepackage[czech]{babel}
\usepackage{amsmath, amsthm}
\usepackage{enumerate}

%% algoritmy
%{{{ Latex definitions
% ALGORITHM typesetting
\usepackage[noline,boxruled,longend,linesnumbered,czech]{algorithm2e}
%\SetKwInOut{Input}{vstup}\SetKwInOut{Output}{výstup}
\SetKwInput{Input}{vstup}
\SetKwInput{Output}{výstup}
\newcommand{\donothing}[1]{#1}
\SetArgSty{donothing}
\SetCommentSty{emph}
%\IncMargin{1em}
%\setlength{\algomargin}{1em}
\DontPrintSemicolon
\SetKwIF{If}{ElseIf}{Else}{if}{then}{else if}{else}{fi}
\SetKwFor{For}{for}{do}{od}
\SetKwFor{While}{while}{do}{od}

% hack to set the correct width of the algorithm
\usepackage{xpatch}
\xpretocmd{\algorithm}{\hsize=\linewidth}{}{}
\xpretocmd{\function}{\hsize=\linewidth}{}{}
\xpretocmd{\procedure}{\hsize=\linewidth}{}{}
%%/algoritmy
\usepackage{mathtools}
\DeclarePairedDelimiter\floor{\lfloor}{\rfloor}

%%% ZDE SADA, UKOL CISLO X, JMENO, UCO
\setexercise{1}
\setassignment{2}
\setstudenta{Karel Kubíček}
\setucoa{408351}
\setstudentb{Henrich Lauko}
\setucob{410438}


\begin{document}

\begin{enumerate}
	\item Posloupnost $n$ operací \textsc{Insert} a \textsc{Min-All} má složitost $\mathcal{O}(n)$.\\
	Neplatí. Můžeme vložit $n/2$ stejných prvků, následně můžeme volat \textsc{Min-All}, která nesmaže žádný prvek, tím pádem můžeme volat tuto operaci $n/2$-krát vždy se složitostí $\mathcal{O}(n)$, tedy $n/2 \cdot \mathcal{O}(n) = \mathcal{O}(n^{2})$. 
	
	\item Posloupnost $n$ operací \textsc{Insert} a \textsc{Min-one} má složitost $\mathcal{O}(n)$.\\
	Platí. Dokážeme pomocí metody účtů. Přiřadíme nasledující ceny operacím:
	\begin{center}
	\begin{tabular}{ l | c | c }
		operace 			&	reálni cena ($c_{i}$)	& 	amortizovaná cena ($\widehat{c_{i}}$) \\ \hline \hline
		\textsc{Insert} 	& 	1						&	2 \\
		\textsc{Min-One}	&	$délka\ seznamu - 1$					&	0 \\ \hline 
	\end{tabular}
	\end{center}
	
	Při \textsc{Insert} vložíme na účet 2 kredity, jeden znich slouži na zaplacení operace \textsc{Insert} a druhej na zaplacení smazáni sama sebe při operaci \textsc{Min-One}. Keďže v každém momentu bude mýt každý prvek jeden kredit, tak účet nemúže jet do mísunu. Tudiž amortizovaná cena může být nanejvýš $2n$ kreditů, kde $n$ je počet operací. Teda pro $n$ operací je amortizovaná složitost $\mathcal{O}(n)$.
	
	\item Posloupnost $n$ operací \textsc{Insert} a \textsc{Delete} má složitost $\mathcal{O}(n)$.\\
	Neplatí. Můžeme vložit $n/2$ stejných prvků $x$, následně můžeme volat \textsc{Delete}($S,x$), která nesmaže žádný prvek, tím pádem můžeme volat tuto operaci $n/2$-krát vždy se složitostí $\mathcal{O}(n)$, tedy $n/2 \cdot \mathcal{O}(n) = \mathcal{O}(n^{2})$. 
	
	\item Posloupnost $n$ operací \textsc{Insert} a \textsc{Delete} taková, že při každém volání se operace \textsc{Delete} volá s jiným parametrem $i$ má složitost $\mathcal{O}(n)$.
	Platí. Dokážeme pomocí metody účtů. Přiřadíme nasledující ceny operacím:
	
	\begin{center}
	\begin{tabular}{ l | c | c }
		operace 			&	reálni cena ($c_{i}$)	& 	amortizovaná cena ($\widehat{c_{i}}$) \\ \hline \hline
		\textsc{Insert} 	& 	1						&	2 \\
		\textsc{Delete}		&	$délka\ seznamu - počet\ prvků\ i$					&	0 \\ \hline 
	\end{tabular}
	\end{center}
	
	Při \textsc{Insert} vložíme na účet 2 kredity, jeden znich slouži na zaplacení operace \textsc{Insert} a druhej na zaplacení smazáni sama sebe při operaci \textsc{Delete}. Keďže v každém momentu bude mýt každý prvek jeden kredit, tak účet nemúže jet do mísunu. Při operaci \textsc{Delete} se smaže jen $délka\ seznamu - počet\ prvků\ i$ prvků, teda počet prvků $i$ musí zostat s 1 kreditem Tudiž amortizovaná cena může být nanejvýš $2n$ kreditů, kde $n$ je počet operací. Teda pro $n$ operací je amortizovaná složitost $\mathcal{O}(n)$.
	
\end{enumerate}

\end{document}


















