\documentclass[12pt]{iv003}
\usepackage[T1]{fontenc}
\usepackage[utf8]{inputenc}
\usepackage{lmodern}
\usepackage[czech]{babel}
\usepackage{amsmath, amsthm}
\usepackage{enumerate}

%% algoritmy
%{{{ Latex definitions
% ALGORITHM typesetting
\usepackage[noline,boxruled,longend,linesnumbered,czech]{algorithm2e}
%\SetKwInOut{Input}{vstup}\SetKwInOut{Output}{výstup}
\SetKwInput{Input}{vstup}
\SetKwInput{Output}{výstup}
\newcommand{\donothing}[1]{#1}
\SetArgSty{donothing}
\SetCommentSty{emph}
%\IncMargin{1em}
%\setlength{\algomargin}{1em}
\DontPrintSemicolon
\SetKwIF{If}{ElseIf}{Else}{if}{then}{else if}{else}{fi}
\SetKwFor{For}{for}{do}{od}
\SetKwFor{While}{while}{do}{od}

% hack to set the correct width of the algorithm
\usepackage{xpatch}
\xpretocmd{\algorithm}{\hsize=\linewidth}{}{}
\xpretocmd{\function}{\hsize=\linewidth}{}{}
\xpretocmd{\procedure}{\hsize=\linewidth}{}{}
%%/algoritmy
\usepackage{mathtools}
\DeclarePairedDelimiter\floor{\lfloor}{\rfloor}

%%% ZDE SADA, UKOL CISLO X, JMENO, UCO
\setexercise{1}
\setassignment{2}
\setstudenta{Karel Kubíček}
\setucoa{408351}
\setstudentb{Henrich Lauko}
\setucob{410438}


\begin{document}
\begin{enumerate}
	\item 
	\item Algoritmus, který potřebujeme, je algoritmus hledání těžiště. To lze provést například takto:\\
	
	\begin{enumerate}
		\item Všem bodům přiřadíme váhu 1.
		\item Procházíme body, vezmeme dvojici a vypočteme její vážený průměr. Vytvoříme tedy nový bod na pozici váženého průměru s váhou rovnou součtu 2 průměrovaných bodů a původní body smažeme.
		\item Opakujeme 2. krok tak dlouho, dokud nám nezůstane jen jeden jediný bod, který bude těžištěm všech předchozích bodů. Tento bod tedy minimalizuje zadanou sumu.
	\end{enumerate}
	Konečnost algoritmu je zřejmá, v každé iteraci se celkový počet bodů zmenší o 1. Časová složitost je lineární k počtu bodů. 1. krok projde včechny body, je tedy proveden v lineárním čase. 2. krok je proveden v konstantním čase, počet opakování je roven počátečnímu počtu bodů bez jedné.\\
	Algoritmus funguje v libovolné dimenzi, jen se mění vzorec pro počítání průměru ze 2 bodů.
\end{enumerate}
\end{document}


















