\documentclass[12pt]{iv003}
\usepackage[T1]{fontenc}
\usepackage[utf8]{inputenc}
\usepackage{lmodern}
\usepackage[czech]{babel}
\usepackage{amsmath, amsthm}
\usepackage{enumerate}

%% algoritmy
%{{{ Latex definitions
% ALGORITHM typesetting
\usepackage[noline,boxruled,longend,linesnumbered,czech]{algorithm2e}
%\SetKwInOut{Input}{vstup}\SetKwInOut{Output}{výstup}
\SetKwInput{Input}{vstup}
\SetKwInput{Output}{výstup}
\newcommand{\donothing}[1]{#1}
\SetArgSty{donothing}
\SetCommentSty{emph}
%\IncMargin{1em}
%\setlength{\algomargin}{1em}
\DontPrintSemicolon
\SetKwIF{If}{ElseIf}{Else}{if}{then}{else if}{else}{fi}
\SetKwFor{For}{for}{do}{od}
\SetKwFor{While}{while}{do}{od}

% hack to set the correct width of the algorithm
\usepackage{xpatch}
\xpretocmd{\algorithm}{\hsize=\linewidth}{}{}
\xpretocmd{\function}{\hsize=\linewidth}{}{}
\xpretocmd{\procedure}{\hsize=\linewidth}{}{}
%%/algoritmy

%%% ZDE SADA, UKOL CISLO X, JMENO, UCO
\setexercise{1}
\setassignment{2}
\setstudenta{Karel Kubíček}
\setucoa{408351}
\setstudentb{Henrich Lauko}
\setucob{410438}


\begin{document}
\begin{small}
\begin{procedure}[H]
\SetKwFunction{fffindEso}{findEso}
    \caption{findEso($A, fromX, toX, fromY, toY$) }
    \Input{čtvercová matice $A$ vymezená v intervalu $x \in \langle fromX, toX\rangle, y \in \langle fromY, toY\rangle$}
    \Output{souřadnice esa v matici, tedy prvek, který je větší, než jeho sousedé}
    \If{$toX = fromX$}{
		\Return{$fromX, fromY$} \tcp{nalezeno eso, vracím souřadnice}
    }
    $halfX \leftarrow (toX - fromX)/2$ \;
    $halfY \leftarrow (toY - fromY)/2$ \;
    \tcp{maxX a maxY jsou souřadnice maxima z prvků kolem horizontálního a vertikálního středu}
    $maxX \leftarrow halfX$ \tcp{pro hledání maxima začnu nastavením souřadnic na 1. prvek}
    $maxY \leftarrow fromY$ \;
    \For{$x \leftarrow fromX$ \KwTo $toX$}{
		\If{$A[maxX][maxY] < A[x][halfY]$}{
			$maxX \leftarrow x$ \tcp{zde procházím sloupec vlevo od středu}
			$maxY \leftarrow halfY$ \;
		}
		\If{$A[maxX][maxY] < A[x][halfY+1]$}{
			$maxX \leftarrow x$ \tcp{zde procházím sloupec vpravo od středu}
			$maxY \leftarrow halfY+1$ \;
		}
    }
    \For{$y \leftarrow fromY$ \KwTo $toY$}{
		\If{$A[maxX][maxY] < A[halfX][y]$}{
			$maxX \leftarrow halfX$ \tcp{zde procházím řádek vlevo od středu}
			$maxY \leftarrow y$ \;
		}
		\If{$A[maxX][maxY] < A[halfX+1][y]$}{
			$maxX \leftarrow halfX+1$ \tcp{zde procházím řádek vpravo od středu}
			$maxY \leftarrow y$ \;
		}
    }
    \uIf{maxX <= halfX} {
		\uIf{maxY <= halfY} {
			\Return{\fffindEso($A, fromX, halfX, fromY, halfY$)} 
		}
		\uElse {
			\Return{\fffindEso($A, fromX, halfX, halfY+1, toY$)} 
		}
    }
    \uElse {
		\uIf{maxY <= halfY} {
			\Return{\fffindEso($A, halfX+1, toX, fromY, halfY$)} 
		}
		\uElse {
			\Return{\fffindEso($A, halfX+1, toX, halfY+1, toY$)} 
		}
    }
\end{procedure}
\end{small}
\paragraph{Pojmy:} abych příliš často nemusel formálně popisovat množinu prvků matice, která se skládá z prvků ve sloupcích odpovídajících $halfX$ a $halfX+1$ v rozsahu $y \in <fromY, toY>$ a řádkům odpovídajících $halfY$ a $halfY+1$ v rozsahu $x \in <fromX, toX>$, tak pro tuto množinu prvků zavádím pojem \textit{středový~kříž}.
\paragraph{Popis algoritmu:} tento rekurzivní algoritmus je založen na tom, že se po každém průchodu změnší část matice, se kterou pracuje na čtvrtinu. Volbou maxima ze středového kříže se rozhoduje, kterou ze 4 možných čtvrtin má vybrat. Nalezení maxima zaručuje, že v dané čtvrtině maximum musí být (formálně rozebráno níže), rekurzivně se tedy algoritmus zanoří do dané čtvrtiny. Eso tedy nemusí být maximum z prvního výběru. Algoritmus by šel upravit, aby místo dělení na čtvrtiny dělil matici na poloviny, střídavě horizontálně a vertikálně, díky čemuž by měl trochu menší složitost (asymptotická složitost by však stále byla $\mathcal{O}(n)$), cenou by však byl mnohem složitější zápis. Při volání rekurze je nutno kontrolovat, aby se volala se čtvercovou maticí, což v pseudokódu není zapsáno (z důvodu složitého zápisu). Pokud by měla zvolená čtvrtina jinou šířku než délku, pak je nutno rekurzi předat větší matici (tak, aby jeden řádek navíc zaručil čtvercovost).
\paragraph{Korektnost:} algoritmus je totálně korektní, pokud skončí a výstup splňuje výstupní podmínku, kterou je v našem případě, že nalezený prvek je eso. Konečnost je zřejmá, jelikož rozměry matice $n$ po každém kroce klesnou minimálně na $\lceil \frac{n}{2} \rceil$, přičemž pro $n = 1$ (k čemuž algoritmus konverguje) algoritmus skončí.\\
Abychom mohli dokázat, že je algoritmus parciálně korektní, je nutno dokázat invariant: v každém rekurzivním volání existuje v daném intervalu matice alespoň 1 eso.\\
Již ze zadání vyplívá, že při prvním volání s celou vstupní maticí musí invariant platit. To je zaručeno unikátností všech prvků.\\
Předpokládejme platnost invariantu pro $m$-té rekurzivní volání. 
%%Bez újmy na obecnosti si můžeme říct, že algoritmus bude pracovat se submaticí v levém horním rohu.\\
V $m+1$-ním rekurzivním volání z předchozího tvrzení víme, že v aktuální submatici se nachází eso. %%Nyní si situaci můžeme rozdělit na 2 možnosti.\\
%%1. maximum z minulého rekurzivního volání je jediným esem v aktuální submatici. Pak od tohoto esa musí existovat klesající posloupnost čísel (nemusí být jen jedna), která vede k maximu ze středového kříže aktuální submatice. Jediným prvkem této posloupnosti, který se nachází na středovém kříži může být naše nalezené maximum. Tato posloupnost se tedy nachází v pouze jedné čtvrtině z aktuální submatice, tato čtvrtina odpovídá čtvrtině, v níž se nachází nalezené maximum a zároveň se jedná o čtvrtinu, ve které se nachází eso. Po dalším rekurzivním volání se tedy invariant nemění.
%%2. Es je více, nebo dané jediné eso není maximem z původního rekurzivního volání.
Pak platí, že od maxima ze středového kříže vede rostoucí posloupnost k esu a jediným prvkem této posloupnosti, který se nachází na středovém kříži, je právě nalezené maximum. Pokud by toto tvrzení neplatilo a posloupnost by měla další prvky ležící na středovém kříži, pak by díky tranzitivitě relace \textit{větší než} musel být daný jiný prvek posloupnosti ležící na kříži větší než nalezené maximum a tím pádem by měl být sám maximem.
\paragraph{Asymptotická časová složitost} algoritmu je $\Theta(n)$. To lze odvodit z rekurentní rovnice, která popisuje složitost každého volání rekurze a vypadá takto:\\
$$T(n) = T(\lceil \frac{n}{2} \rceil) + \mathcal{O}(n)$$
$$T(1) = 1$$
Tato rovnice říká, že při každém rekurzivním volání se šířka matice zmenšuje na polovinu (zaokrouhleno nahoru), a před tím se provede $4n$ porovnání při hledání maxima ve středovém kříži. Dalších operací je konstantní počet, takže platí, že $4n + k \in \mathcal{O}(n)$.
Horní hranici získáme pomocí master theoremu, přičemž naše rekurentní rovnice nabývá hodnot $a = 1$, $b = 2$ a $d = 1$, tedy $a < b^{d} \Rightarrow T(n) \in \mathcal{O}(n)$. Složitost máme omezenou také zdola, hned v prvním volání provádíme $4n$ porovnání, tím pádem náš algoritmus patří do $\Theta(n)$.
\end{document}
