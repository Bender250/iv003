\documentclass[12pt]{iv003}
\usepackage[T1]{fontenc}
\usepackage[utf8]{inputenc}
\usepackage{lmodern}
\usepackage[czech]{babel}
\usepackage{amsmath, amsthm}
\usepackage{enumerate}

%% algoritmy
%{{{ Latex definitions
% ALGORITHM typesetting
\usepackage[noline,boxruled,longend,linesnumbered,czech]{algorithm2e}
%\SetKwInOut{Input}{vstup}\SetKwInOut{Output}{výstup}
\SetKwInput{Input}{vstup}
\SetKwInput{Output}{výstup}
\newcommand{\donothing}[1]{#1}
\SetArgSty{donothing}
\SetCommentSty{emph}
%\IncMargin{1em}
%\setlength{\algomargin}{1em}
\DontPrintSemicolon
\SetKwIF{If}{ElseIf}{Else}{if}{then}{else if}{else}{fi}
\SetKwFor{For}{for}{do}{od}
\SetKwFor{While}{while}{do}{od}
\SetFuncSty{textsc} \SetProcNameSty{textsc}

% hack to set the correct width of the algorithm
\usepackage{xpatch}
\xpretocmd{\algorithm}{\hsize=\linewidth}{}{}
\xpretocmd{\function}{\hsize=\linewidth}{}{}
\xpretocmd{\procedure}{\hsize=\linewidth}{}{}
%%/algoritmy

%%% ZDE SADA, UKOL CISLO X, JMENO, UCO
\setexercise{3}
\setassignment{2}
\setstudenta{Karel Kubíček}
\setucoa{408351}
\setstudentb{Martin Hanžl}
\setucob{410497}

\begin{document}

\SetKwFunction{fftopologicalSort}{topologicalSort}
\SetKwFunction{ffmax}{max}

\paragraph{Popis algoritmu:} pro konstrukci tohoto algoritmu jsme využili principu dynamického programování. Konkrétně místo opakovaného počítání výsledků v rekurzi si mezivýsledky pamatujeme pomocí memoizace. K tomu nám složí matice rozměrů $(|z| + 1) \times (|z| + 1)$, na pozici $[i,j]$ uchovává boolovskou hodnotu, vyjadřující, zdali se z $i$ znaků repretice $x$ a $j$ symbolů repetice $y$ dá vytvořit spojení $x$ a $y$.\\
Matice je schodovitá, výstup funkce záleží na hodnotách na diagonále, což jsou body se vzdáleností odpovídající celému slovu $z$, kde cesta maticí popisuje, jak spojení vypadá.\\
Matici číslujeme od 0, zatímco pozici ve slovech od -1.\\

{\small \begin{procedure}[H]
\SetKwFunction{ffisconnection}{isConnection}
	\caption{isConnection($x, y, z$)}
	\Input{o slovu $z$, zjišťujeme, zdali je spojením slov $x$ a $y$}
	\Output{boolovská hodnota, zdali $z$ je spojením $x$ a $y$}
	\tcp{matice pro memoizaci}
	$matrix \leftarrow $ matice $(|z| + 1) \times (|z| + 1)$ \;
	$matrix[0][0] \leftarrow $ \textsc{True} \tcp{prázdné slovo z $x$ a $y$ umíme vytvořit}
	\tcp{inicializace okraje matice, tedy pro slova, které jsou jen repeticí jednoho ze slov}
	\For{$i \leftarrow 1$ \KwTo $|z|$}{
			$matrix[0][i] \leftarrow$ $(matrix[0][i - 1] \wedge x_{i - 1 \mod |x|} = z_{i - 1})$ \tcp{přechod pod slovem $x$}
			$matrix[i][0] \leftarrow$ $(matrix[i -1][0] \wedge y_{i - 1 \mod |y|} = z_{i - 1})$ \tcp{přechod pod slovem $y$}
	}
	\If{$matrix[0][|z|] \vee matrix[|z|][0]$}{
		\Return{\textsc{True}} \tcp{z je repeticí jen jednoho slova}
	}
	\tcp{memoizace}
	\For{$i \leftarrow 1$ \KwTo $|z|$} {
		\For{$j \leftarrow 1$ \KwTo $|z|-i$} { %+1 ne?
			\uIf{$(matrix[i][j - 1] \wedge x_{j - 1 \mod |x|} = z_{i + j - 1})$}{
				$matrix[i][j] \leftarrow$ \textsc{True} \tcp{existuje přechod pod aktuálním znakem ze slova $x$}
			}\uElseIf{$(matrix[i - 1][j] \wedge y_{i - 1 \mod |y|} = z_{i + j - 1})$}{
				$matrix[i][j] \leftarrow$ \textsc{True} \tcp{existuje přechod pod aktuálním znakem ze slova $y$}
			}\uElse{
				$matrix[i][j] \leftarrow$ \textsc{False} \tcp{neexistuje přechod pod žádným znakem}
			}
		}
		\uIf{$matrix[i][|z|-i]$} {
			\Return{\textsc{True}}
		}
	}
	\Return{\textsc{False}}
\end{procedure}}
\paragraph{Asymptotická časová složitost:} inicializace matice $(|z| + 1) \times (|z| + 1)$ je v $\mathcal{O}|z|^{2}$. První for cyklus prochází znovu délku slova $z$. Až 2 zanořené for cykly od řádku 10 jsou zase v kvadratickém čase. Celkově je tedy složitost kvadratická.
\paragraph{Korektnost:} Algoritmus je korektní, pokud na vstupech splňujících vstupní podmínku skončí a zároveň je parciálně korektní.\\
Konečnost algoritmu je implikována použitím pouze for cyklů, které iterují nad délkou slova $z$. Jelikož se proměnné těchto for cyklů mění jen ve forcyklech a to vždy vzhůru, algoritmus se nemá kde zacyklit a skončí.\\
Parciální korektnost dokážeme indukcí přes obsah matice.\\
Pro indexy $[0,0]$ je zřejmé, že prázdné slovo z $x$ a $y$ poskládat lze.
Očekávejme, že algoritmus platí pro slovo délky $k$, pak algoritmus platí i pro slovo délky $k+1$. Z indukčního předpokladu víme, že máme napočítanou matici po diagonálu ve vzdálenosti $k$ od $[0,0]$. Na této diagonále jsou informace o možnosti složit slovo délky $k$. Tuto matici můžeme výpočtem doplnit o novou diagonálu ve vzdálenosti $k+1$. Pozici v nové diagonále určíme podle levé a dolní pozice. V případě, že jsou obě \textsc{False}, pak ani nová pozice nemůže být \textsc{True}, což odpovídá tomu, že jakmile nejsme schopni spojit nějaká slova, pak nejsme schopni spojit ani jejich slova prodloužené o nějaký symbol. V případě alespoň jednoho z levého, nebo dolního souseda s hodnotou \textsc{True} chceme do slova vzniklého spojením přidat znak, který odpovídá $n$-tému znaku v repetici slova, ze kterého vybíráme. To je dáno pohybem v matici, pohyb doprava znamená výběr znaku ze repetice slova $x$ a obdobně pro $y$. Pokud další přidaný znak odpovídá $k+1$. symbolu slova $z$, pak přiřadíme na zkoumanou pozici v matici hodnotu \textsc{True}. Algoritmus je tedy korektní

\end{document}
