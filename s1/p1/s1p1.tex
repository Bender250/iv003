\documentclass[12pt]{iv003}

\setassignment{1}
\setexercise{1}

% BASIC packages
\usepackage[utf8]{inputenc}
\usepackage[czech]{babel}
\usepackage[T1]{fontenc}
\usepackage{microtype}
\usepackage{lmodern}
%\usepackage[margin=1in]{geometry}
\usepackage{amssymb}
\usepackage{ifthen}
\usepackage[all]{xy}
\usepackage{fancyvrb}
\usepackage{enumerate}

\usepackage{pgfplots}

% ALGORITHM typesetting
\usepackage[noline,boxruled,longend,linesnumbered,czech]{algorithm2e}
\SetKwInput{Input}{vstup}
\SetKwInput{Output}{výstup}
\newcommand{\donothing}[1]{#1}
\SetArgSty{donothing}
\SetCommentSty{emph}
\DontPrintSemicolon
\SetKwIF{If}{ElseIf}{Else}{if}{then}{else if}{else}{fi}
\SetKwFor{For}{for}{do}{od}
\SetKwFor{While}{while}{do}{od}

\usepackage{xpatch}
\xpretocmd{\algorithm}{\hsize=\linewidth}{}{}
\xpretocmd{\function}{\hsize=\linewidth}{}{}
\xpretocmd{\procedure}{\hsize=\linewidth}{}{}

%%Set variables
\setassignment{1}
\setexercise{1}

\setstudentI{Henrich Lauko}
\setucoI{410438}

\setstudentII{Martin Hanžl}
\setucoII{410497}

\begin{document}

\begin{function}[H]
	\caption{WeightedMedian($Array, n, w$)}
	\Input{pole $Array$ je dĺžky $n$, obsahuje dvojice ($x_i,w_i$)}
	\Output{optimálny prvek $x_k$}
	
	$median_{x} \leftarrow$ median($Array$) \tcp{nájde median z hodnôt $\{ x_{1},..,x_{n} \}$}
	$Array_{low} \leftarrow \emptyset$ \;
	$Array_{high} \leftarrow \emptyset$ \;
	$w_{low} \leftarrow 0$ \;
	
	\If{$n = 1$} {
		\Return{$Array[1]$}
	}
	\For{$i \leftarrow 1$ \KwTo $n$}{
		\eIf{$Array[i] < median_{x}$}{
			$w_{low} \leftarrow w_{low} + w_{i}$ \;
			Append $Array[i]$ to $Array_{low}$ \;
		}{
			Appent $Array[i]$ to $Array_{high}$ \;
		}
	
	}
	\eIf{$w + w_{low} > \frac{1}{2}$}{
			WeightedMedian($Array_{low}, |Array_{low}|, w$) \;
	}{
			WeightedMedian($Array_{high}, |Array_{high}|, w_{low}$)
	}
\end{function}

Algoritmus využíva znalosť výpočtu mediánu v $\theta(n)$, dokázaneho na prednáške.
Základna stratégia algoritmu je podobná binárnemu vyhľadávaniu: algoritmus spočíta
medián a rekurzívne pokračuje na časti, kde sa nachádza vážený medián.

Inicilizácia algoritmu je WeightedMedian($Array, |Array|, 0$), kde $|Array|$ udáva počet prvkov poľa. V behu algoritmu, $Array$ je pole obsahujúce $median$ pôvodneho vstupu a $w$ je suma váh prvkov z celkového vstupu algoritmu, ktoré su menšie ako všetky prvky v poli $Array$. 
$Array_{low}$ obsahuje všetky prvky $x_{i}$ menšie ako $median$ z prvkov $Array$ a $Array_{high}$ obsahuje všetky prvky $x_{i}$ väčšie alebo rovné ako $median$ z prvkov $Array$, váha $w_{low}$ je súčet váh všetkých prvkov z $Array_{low}$.

\paragraph{Korektnosť:}
Pre dokázanie korektnosti ukážeme, že vážený medián počiatočnej postupnosti(optimálny prvok $x_{k}$) je vždy v poli rekurzívneho volania a $w$ je suma všetkých váh prvkov $x_{i}$ menších ako všetky prvky v poli $Array$. Pre počiatočný stav podmienka triviálne platí. Pre dokázania, že podmienka platí v každom rekurzívnom volaní použijeme indukciu. Indukčný predpoklad bude, že naša kladená podmienka platí. Môžu nám nastať dva prípady:
\begin{enumerate}
	\item $w + w_{low} > \frac{1}{2}$ Keďže $x_{k}$ sa nachádza v $Array$ v 16. riadku algoritmu vieme, že $x_{k}$ sa nachádza buďto v $Array_{low}$ alebo v $Array_{high}$. Vzhľadom na to, že váha všetkých $x_{i}$ menších než prvky v $Array_{high}$  je väčšia než $\frac{1}{2}$, potom podľa definície vážený medián
	nemôže byť v poli $Array_{high}$, takže musí byť v poli $Array_{low}$. Pole $Array_{low}$ obsahuje všetky najmenšie prvky z pola $Array$, takže $w$ sa nemení a splňuje podmienku.
	\item $w + w_{low} \leq \frac{1}{2}$ potom na riadku 16. musí byť $x_{k}$ v $Array_{high}$. Všetky prvky $Array_{high}$ sú väčšie ako všetky prvky v $Array_{low}$, takže celková váha všetkých prvkov menších ako prvky v $Array_{high}$ je $w + w_{low}$. Teda predpoklad je splnený.

\end{enumerate}

Takže podľa indukcie je podmienka vždy pravdivá. Algoritmus vždy skončí, lebo
pole $Array$ sa každým rekurzivním volaním zmenší. Keďže je algoritmus konečný
aj parciálne korektný tak je aj korektný.

\paragraph{Analýza zložitosti:}
Algoritmus beží v čase $\theta(n)$. Výpočete $medianu$ je v $\theta(n)$ a rozdelenie $Array$ do $Arrya_{low}$ a $Array_{high}$ tiež v $\theta(n)$. Teda zložitosť jedného rekurzívneho volania je $\theta(n) + \theta(n) = \theta(n)$.
Každé rekurzívne volanie zmeší velkosť vstupu $n$ na $\frac{n}{2}$. Rekurentná rovnica
bude $T(n) = T(n/2) +\theta(n) = \theta(n)$.
\end{document}
