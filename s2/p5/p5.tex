\documentclass[12pt]{iv003}

% BASIC packages
\usepackage[utf8]{inputenc}
\usepackage[czech]{babel}
\usepackage[T1]{fontenc}
\usepackage{microtype}
\usepackage{lmodern}
%\usepackage[margin=1in]{geometry}
\usepackage{amssymb}
\usepackage{ifthen}
\usepackage[all]{xy}
\usepackage{fancyvrb}
\usepackage{enumerate}
\usepackage{verbatim}
\usepackage{pdfpages}
\usepackage{amsmath}

\usepackage{pgfplots}

% ALGORITHM typesetting
\usepackage[noline,boxruled,longend,linesnumbered,czech]{algorithm2e}
\SetKwInput{Input}{vstup}
\SetKwInput{Output}{výstup}
\newcommand{\donothing}[1]{#1}
\SetArgSty{donothing}
\SetCommentSty{emph}
\DontPrintSemicolon
\SetKwIF{If}{ElseIf}{Else}{if}{then}{else if}{else}{fi}
\SetKwFor{For}{for}{do}{od}
\SetKwFor{While}{while}{do}{od}
\SetKwFor{Foreach}{foreach}{do}{od}

\usepackage{xpatch}
\xpretocmd{\algorithm}{\hsize=\linewidth}{}{}
\xpretocmd{\function}{\hsize=\linewidth}{}{}
\xpretocmd{\procedure}{\hsize=\linewidth}{}{}

%%Set variables
\setassignment{2}
\setexercise{5}

\setstudentI{Henrich Lauko}
\setucoI{410438}

\setstudentII{Karel Kubíček}
\setucoII{408351}

\begin{document}
\paragraph{Prvý problém} Algoritmus vždy nájde optimálne riešenie. Dokážeme, že algoritmus vždy nájde také riešenie, že neexistuje iné, ktoré by malo menšiu penalizáciu.\\
Majme $k$ ako penalizáciu nášho riešenia. Pre penalizáciu jednoriadkového riešenia tvrdenie platí.\\
Pre viac riadkové riešenie:\\
Predpokladajme, že algorimus nájde riešenie s penalizáciou $k$ kde ($r_{1},\dots,r_{i}$) a $r_{i}$ vyjadruje penalizáciu na danom riadku.
Uvažujme iné riešenie $k'$ s penalizáciami ($s_{1},\dots,s_{j}$). Podľa nášho hladového algoritmu vieme, že $s_{1} \geq r_{1}$, lebo sme $r_{1}$ zobrali ako najmenšiu možnú penalizáciu pre prvý riadok. Modifikáciou riešenia ($s_{1},\dots,s_{j}$) tak, že presunieme slová z druhé riadka tak, aby sa $s'_{1} = r_{1}$ a odobraním tejto penalizácie dostávame problém s penalizáciami  $(s'_{2},\dots,s'_{l})$ kde $k'' \leq k' - (r_{1} - s_{1})$  , ktorý je o jeden riadok kratši od pôvodneho problému, teda jeho riešenením je ($r_{2},\dots,r_{i}$). Induktívne $k - r_{1}  \leq k'' \leq k' - r_{1}$, takže platí aj $k \leq k'$. Algoritmus je teda optimálny.

\paragraph{Druhý problém} Algoritmus nenájde optimálne riešenie, protipríklad:\\
Majme $L = 8$ a nasledujúcu vetu "$AAA$ $BB$ $CCC$ $DDD$ $EE$ $FFFFF$".
Potom hladový algoritmus rozdelí vetu nasledovne:
\begin{center}
\begin{tabular}{lc|c|c}
	veta: 			& $AAA$ $BB$ $CCC$ 	& $DDD$ $EE$ 	& $FFFFF$ \\
	penalizácia ($L - K^{2}$):	&	0				&	9			&	9	\\
\end{tabular}
\end{center}
Ale optimálne rozdelenie je:
\begin{center}
\begin{tabular}{lc|c|c}
	veta: 						& $AAA$ $BB$ 	& $CCC$ $DDD$ 	& $EE$ $FFFFF$ \\
	penalizácia ($L - K^{2}$):	&	9			&	4			&	1	\\
\end{tabular}
\end{center}
Keďže súčet penalizácií hladového algoritmu (18) je väčší ako súčet penalizácií optimálneho riešenia (14), tak algoritmus nemusí nájsť optimálne riešenie pre tento problém.

\paragraph{Tretí problém}Algoritmus nenájde optimálne riešenie, protipríklad:\\
Majme $L = 4$ a nasledujúcu vetu "$A$ $A$ $A$ $A$ $BB$".
Potom hladový algoritmus rozdelí vetu nasledovne:
\begin{center}
\begin{tabular}{lc|c}
	veta: 					& $A$ $A$ $A$ $A$ 	& $BB$  \\
	penalizácia ($L - K$):	&	0				&	2	\\
\end{tabular}
\end{center}
Celková penalizácia je maximum z penalizácií teda 2.
Ale optimálne rozdelenie je:
\begin{center}
\begin{tabular}{lc|c}
	veta: 					& $A$ $A$ $A$  	& $A$ $BB$ \\
	penalizácia ($L - K$):	&	1			&	1	\\
\end{tabular}
\end{center}
Pre optimálne rozdelenie je penalizácia 1. Keďže existuje riešenie s menšou penalizáciou ako riešenie hladového algoritmu, tak hladový algoritmus nie je optimálny pre riešenie tohto problému.
\end{document}

