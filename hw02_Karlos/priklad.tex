\documentclass[12pt]{iv003}
\usepackage[T1]{fontenc}
\usepackage[utf8]{inputenc}
\usepackage{lmodern}
\usepackage[czech]{babel}
\usepackage{amsmath, amsthm}
\usepackage{enumerate}

%% algoritmy
%{{{ Latex definitions
% ALGORITHM typesetting
\usepackage[noline,boxruled,longend,linesnumbered,czech]{algorithm2e}
%\SetKwInOut{Input}{vstup}\SetKwInOut{Output}{výstup}
\SetKwInput{Input}{vstup}
\SetKwInput{Output}{výstup}
\newcommand{\donothing}[1]{#1}
\SetArgSty{donothing}
\SetCommentSty{emph}
%\IncMargin{1em}
%\setlength{\algomargin}{1em}
\DontPrintSemicolon
\SetKwIF{If}{ElseIf}{Else}{if}{then}{else if}{else}{fi}
\SetKwFor{For}{for}{do}{od}
\SetKwFor{While}{while}{do}{od}
\SetFuncSty{textsc} \SetProcNameSty{textsc}

% hack to set the correct width of the algorithm
\usepackage{xpatch}
\xpretocmd{\algorithm}{\hsize=\linewidth}{}{}
\xpretocmd{\function}{\hsize=\linewidth}{}{}
\xpretocmd{\procedure}{\hsize=\linewidth}{}{}
%%/algoritmy

%%% ZDE SADA, UKOL CISLO X, JMENO, UCO
\setexercise{2}
\setassignment{1}
\setstudenta{Karel Kubíček}
\setucoa{408351}
\setstudentb{Henrich Lauko}
\setucob{410438}

\begin{document}

\SetKwFunction{ffmin}{Min} %v pripade stejnych hodnot bere prvni
\SetKwFunction{fftakePair}{takePair} %po konci je vse NaN

{\small \begin{procedure}[H]
\SetKwFunction{ffcomputeSiluete}{computeSiluete}
    \caption{computeSiluete($Input, from, to$) }
    \Input{pole uspořádaných trojic $Input$, využíváme v zanoření rozsah od $from$ do $to$}
    \Output{vypočtená silueta v poli uspořádané $n$-tice $Output$}
    \If{$to = from$}{
		\Return{$Input[from]$} \tcp{rekurzivní zarážka, vrací pole 3 čísel}
    }
    $half \leftarrow (to + from)/2$ \;
    $Output[0] \leftarrow$ \ffcomputeSiluete($Input, from, half$) \;
    $Output[1] \leftarrow$ \ffcomputeSiluete($Input, half+1, to$) \;
	\tcp{nechť u těchto polí můžeme pomocí funkce \fftakePair($i$) vzít $i$-tou dvojici, přičemž poslední číslo utvoří dvojici s \textsc{NaN}}
	dvojice souřadnic $first(x,y) \leftarrow$ \ffmin($Output[0]$.\fftakePair(0), $Output[1]$.\fftakePair(0)) \tcp{vybere pole s minimem $x$, z tohoto pole vezme i $y$}
	\uIf{$first \in Output[0]$} {
		$chosen \leftarrow 0$
	}
	\uElse {
		$chosen \leftarrow 1$
	}
	$counter[chosen] \leftarrow 1$ \;
	$counter[!chosen] \leftarrow 0$ \;
	$NewOutput \leftarrow$ prázdné pole \;
    \While{$first.y \not=$\textsc{NaN}} {
		$second(x,y) \leftarrow Output[!chosen].$ \fftakePair($counter[!chosen]$) \;
		%\While{$second.y < first.y~\vee~second.x < Output[!chosen].$\fftakePair($counter[chosen]$).$x$} {
			%$second \leftarrow Output[!chosen].$ \fftakePair($counter[!chosen]$) \;
			%$counter[!chosen] \leftarrow counter[!chosen] + 1$
		%}
		\If{$second.y > first.y$} {
			přidej dvojici $first$ do $NewOutput$ \;
			$first \leftarrow second$ \;
			$second \leftarrow Output[chosen].$ \fftakePair($counter[chosen]$) \;
			$counter[chosen] \leftarrow counter[chosen] + 1$ \;
			$chosen \leftarrow !chosen$ \;
		}
		$following \leftarrow Output[!chosen].$\fftakePair($counter[!chosen]$) \;
		\If{$second.x < folowing.x$} {
			\tcp{\textsc{NaN} je vždy větší než libovolné číslo}
			přidej dvojici $first$ do $NewOutput$ \;
			$first(x,y) \leftarrow$ \ffmin($Output[0]$.\fftakePair($counter[0]$), $Output[1]$.\fftakePair($counter[1]$)) \;
			\uIf{$first \in Output[0]$} {
				$chosen \leftarrow 0$
			}
			\uElse {
				$chosen \leftarrow 1$
			}
			$counter[chosen] \leftarrow counter[chosen] + 1$ \;
		}
    }
    přidej $first.x$ do $NewOutput$ \;
    \Return($NewOutput$)
\end{procedure}}
\paragraph{Popis algoritmu:} samotná myšlenka algoritmu je jednoduchá. Vstup rozdělíme na jednotlivé trojice reprezentující siluetu jedné budovy. V kroku "panuj" pak spojujeme vždy 2 dané siluety. Nejtěžším krokem je spojení. Musíme procházet dvojici siluet a ověřovat, která je v dané pozici vyšší a tu přidáme do výsledné siluety. To se v algoritmu děje od řádku 7.\\
Prvně musíme vybrat siluetu, která má první budovu s menší souřadnicí $x$. Pomocná funkce \ffmin dostane jako vstup 2 dvojice a ty porovná podle první složky - $x$. Poté musíme zjistit, ze kterého pole jsme dvojici vybrali a správně nastavit počítadla vybraných dvojic. Zápisem $!chosen$ myslím "negaci", tedy druhou možnou hodnotu.\\
Nejzajímavější částí algoritmu je cyklus. Ten se zastaví v případě, že jsme již prošli obě pole, což lze testovat třeba tak, že otestujeme, jestli je ve dvojici $first$ číslo. To nám rozšiřuje nároky na funkci \ffmin - ta musí fungovat i pro \textsc{NaN}, v takovém případě vždy volí libovolné číslo. Pro 2 \textsc{NaN} ukládá do dvojice \textsc{NaN}. V Cyklu s každým průchodem posouváme dvojici $second$, kterou testujeme vůči 2 různým věcem. Zaprvé ji porovnáváme s $y$ hodnotou první dvojice, zadruhé testujeme, zdali už jsme se v pomyslném počítadle na ose $x$ neposunuli na další budovu v první siluetě. Do výstupu přidáváme první dvojici za podmínky, že vstoupíme do jednoho z \textsc{if} bloků. První z podmínek v případě splnění prohazuje siluety, ze kterých budeme vybírat, druhá jen posouvá dvojici na další budovu stejné siluety.
\paragraph{Korektnost:} 
\paragraph{Asymptotická časová složitost:} algoritmu je dle zadání $\mathcal{O}(n\cdot\log(n))$. Rekurze se volá $\log_{2}(n)$-krát přičemž v $k$-tém rekurzivním zanoření (první volání bez rekurze odpovídá $k=0$) budu spojovat celkem $2^{k}$ $m$-tic, kde $m$ odpovídá maximálně $\frac{3\cdot 2^{\log_{2}(n)-k}}{2^{k}}$, kde $n$ je počet budov (počet trojic). $m$ jsem odvodil tak, že v nejspodnějším patře spojuji trojice, každé vyšší patro délku $m$-tice maximálně zdvojnásobí. Po úpravě a roznásobení získáme hodnotu $3n$, která odpovídá maximálnímu počtu operací (zanedbáváme konstantní počet operací porovnání) na každé úrovni zanoření rekurze. To tedy spolu s $\log_{2}(n)$ počty zanoření dává $\mathcal{O}(n\cdot\log(n))$
\end{document}
